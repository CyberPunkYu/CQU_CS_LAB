\section{实验设计}

\subsection{数据通路}\label{sub:datapath}
\subsubsection{功能描述}
简易单周期 CPU,支持整数算术逻辑运算,存取,跳转等指令。

\subsubsection{接口定义}

\begin{table}[htp]
\caption{接口定义}\label{tab:signaldef}
\begin{center}
	\begin{tabular}{lllp{6cm}} \hline
	\textbf{信号名} & \textbf{方向} & \textbf{位宽} & \textbf{功能描述}\\ \hline
	writedata		& Output & 32-bit & regfile取出的RD2操作数,写入数据存储器\\ 
	dataadr         & Output & 32-bit & 数值为ALU的结果,当memwrite有效时为写入存储器地址,无效时为读出存储器地址\\
	instr           & Output & 32-bit & 从inst\_ram中读取的32位MIPS指令\\
	pc              & Output & 32-bit & 程序计数器,每个时钟上升沿自增4,指向下一条指令\\
	readdata        & Output & 32-bit & 从数据存储器读出的数据 \\ \hline
	\end{tabular}
\end{center}
\end{table}

\subsubsection{逻辑控制}
\caption{逻辑控制}\label{tab:signaldef}
\begin{center}
	\begin{tabular}{llp{6cm}} \hline
	\textbf{信号名} & \textbf{位宽} & \textbf{功能描述}\\ \hline
	memtoreg	& 1-bit &  多选器控制信号,高电平代表写入寄存器堆的数据为从数据存储器中读取的数据,低电平代表alu结果写入寄存器堆\\
	memtwrite	& 1-bit &  数据存储器写使能信号,高电平有效\\
	pcsrc		& 1-bit &  多选器控制信号,高电平代表条件跳转指令生效,低电平代表pc+4\\
	branch		& 1-bit &  高电平代表该指令为条件跳转指令\\
	alusrc		& 1-bit &  多选器控制信号,高电平代表alu的第二个操作数来自符号拓展后的立即数,低电平代表RD2\\
	regdst		& 1-bit &  多选器控制信号,高电平代表寄存器堆的写地址为ra3,低电平代表ra2\\
	regwrite	& 1-bit &  寄存器堆写使能信号\\
	jump        & 1-bit &  多选器控制信号,高电平代表无条件跳转指令生效,低电平代表不生效\\ \hline
	\end{tabular}
\end{center}
\end{table}