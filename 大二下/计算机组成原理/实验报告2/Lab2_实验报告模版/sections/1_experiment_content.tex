\section{实验内容}
\begin{enumerate}
    \item PC。D触发器结构,用于储存PC(一个周期)。需\textcolor{red}{实现2个输入},分别为\textit{clk, rst},分别连接时钟和复位信号;需\textcolor{red}{实现2个输出},分别为\textit{pc, inst\_ce}, 分别连接指令存储器的\textit{addra, ena}端口。其中\textit{addra}位数依据coe文件中指令数定义;
    \item 加法器。用于计算下一条指令地址,需\textcolor{red}{实现2个输入,1个输出},输入值分别为当前指令地址\textit{PC、32’h4};
    \item Controller。其中包含两部分:
    \begin{enumerate}
        \item main\_decoder。负责判断指令类型,并生成相应的控制信号。需\textcolor{red}{实现1个输入},为指令inst的高6位\textit{op},输出分为2部分,\textcolor{epubblue}{控制信号}有多个,可作为多个输出,也作为一个多位输出,具体参照参考指导书进行设计;\textit{aluop},传输至alu\_decoder,使alu\_decoder配合\textit{inst}低6位\textit{funct},进行ALU模块控制信号的译码。
        \item alu\_decoder。负责ALU模块控制信号的译码。需\textcolor{red}{实现2个输入,1个输出},输入分别为\textit{funct, aluop};输出位\textit{alucontrol}信号。
        \item 除上述两个组件,需设计controller文件调用两个decoder,\textcolor{epubblue}{对应实现\textit{op,funct}输入信号,并传入调用模块;对应实现控制信号及\textit{alucontrol},并连接至调用模块相应端口}。
    \end{enumerate}

    \item 指令存储器。使用Block Memory Generator IP构造。(参考指导书)
    
    \textcolor{red}{ 注意:	Basic中Generate address interface with 32 bits 选项不选中;PortA Options中 Enable Port Type 选择为 Use ENA Pin}
    \item 时钟分频器。将板载100Mhz频率降低为1hz,连接PC、指令存储器时钟信号clk。(参考数字逻辑实验)
    
    \textcolor{red}{注意:Xilinx Clocking Wizard IP可分的最低频率为4.687Mhz,因而只能使用自实现分频模块进行分频}
\end{enumerate}