\section{实验设计}

\subsection{ALU}\label{sub:alu}

\subsubsection{功能描述}
{在optional code的控制下,输入的8位操作数A能在该算数逻辑单元中与内置操作数B进行加、减、与、或、非和SLT的运算。}

\subsubsection{接口定义}
\begin{table}[htp]
\caption{接口定义模版}
\begin{center}
	\begin{tabular}{lllp{6cm}}
	\hline
	\textbf{信号名} & \textbf{方向} & \textbf{位宽} & \textbf{功能描述}\\ \hline
	clk & Input & 1-bit & 时钟信号 \\
	reset & Input & 1-bit & 分时复用重置 \\
	num1 & Input & 8-bit & 操作数A \\
	op & Input & 3-bit & operation code 000表示加法,001表示减法,010表示与,011表示或,100表示非,101表示ALT\\
	seg & Output & 7-bit & 七段数码管选通 \\
	ans & Output & 8-bit & 八个数码管显示运算结果 \\
	\hline
	\end{tabular}
\end{center}
\end{table}

\subsubsection{逻辑控制}{\begin{enumerate}
\item ALU的设计采用always实现组合逻辑。通过op来控制ALU实现不同运算功能。
其中op通过sw[15:13]控制,num1通过sw[7:0]控制。
\item 输入八位num1后,会将该八位操作数无符号拓展为32位,再输入ALU进行运算。
\item 七段数码管使用时序逻辑方法,采取分时复用使运算结果显示到八个七段数码管上。
\end{enumerate}}
\subsection{有阻塞4级8bit全加器}\label{sub:ctl}

\subsubsection{功能描述}
实现 4 级流水线 32bit 全加器,每一级进行 8bit 加法运算,带有各级流水线暂停和刷新。

\newpage
\subsubsection{接口定义}
\begin{table}[htp]
\caption{接口定义}
\begin{center}
	\begin{tabular}{lllp{8cm}}
	\hline
	\textbf{信号名} & \textbf{方向} & \textbf{位宽} & \textbf{功能描述}\\ \hline
	clk    & Input & 1-bit  & 时钟信号 \\
	a  & Input & 32-bit & 操作数A \\
	b  & Input & 32-bit & 操作数B \\
	cin   & Input & 1-bit  & 该加法器初始进位信号 \\
	rst   & Input & 1-bit  & 全局重置信号 \\
	validin & Input & 1-bit  & 一级流水线接受新操作数信号 \\
	out\_allowin & Input & 1-bit & 四级流水线输出最终运算结果信号 \\
	pipeX\_ready\_go & Input & 1-bit & 描述第X级的状态,1表示第X级的处理任务已经完成,可以传给X+1级\\
	fresh\_X & Input & 1-bit & 各级流水线刷新信号 \\
	sum1 & Output & 8-bit  & 一级流水线运算结果\\
	sum2 & Output & 16-bit & 二级流水线运算结果\\
	sum3 & Output & 24-bit & 三级流水线运算结果\\
	sum4 & Output & 32-bit & 四级流水线运算结果(最终结果)\\
	validout & Output & 1-bit & 1代表最终输出结果有效 \\
	cout & Output & 1-bit  & 该加法器是否产生进位\\
	\hline
	\end{tabular}
\end{center}
\end{table}

\subsubsection{逻辑控制}
\begin{enumerate}
    \item 采用时序逻辑控制整个加法器。各级流水线有rst和fresh两个信号(高电平有效)用于各级流水线的重置和刷新。
    \item 我们将32位相加分为四次进行,即每次相加8位,然后用寄存器存储每次相加后的结果、进位情况。其中cout来存储是否进位,sum1、sum2、sum3、sum4存储每次累加后的结果,a/b\_X储存上一级参与运算的操作数。
    \item pipeX\_allowin = !pipeX\_valid || pipeX\_ready\_go && pipeX+1\_allowin,pipe1只有当pipe1的值赋给pipe2或者pipe1数值为无效的时候才允许被赋值。
    \item 在一级流水线中,首先判断rst信号是否有效,有效则置零;其次判断是否允许被赋值,即pipeX\_allowin是否有效,若pipeX\_allowin有效且刷新信号无效,再去判断下一级流水线是否能够接受数据,若可以,则这一级流水线的工作才结束。
\end{enumerate}
\begin{table}[htp]
	\caption{逻辑控制}
	\begin{center}
		\begin{tabular}{lllp{8cm}}
		\hline
		\textbf{信号名}  & \textbf{位宽} & \textbf{功能描述}\\ \hline
		carry\_X & 1-bit & 各级流水线进位情况 \\
		pipeX\_valid & 1-bit & 当前级是否存在有效的数据,高有效\\
		pipeX\_allowin & 1-bit & 第X级传给第X-1级的状态,是否可以接收上一级的数据 \\
		pipeX\_to\_pipeX+1\_valid & 1-bit & 从第X级传递给第X+1级,1表示下一时刻第X级有数据传递给第X+1级 \\
		\hline
		\end{tabular}
	\end{center}
	\end{table}