\newpage
\section{实验过程记录}
\subsection{ALU}
\begin{enumerate}
    \item 使用组合逻辑实现ALU,其中包括加、减、与、或、非、SLT六种运算。在仿真和上板过程中均验证ALU设计功能正常。
\end{enumerate}

\subsection{阻塞流水线加法器}
\begin{enumerate}
    \item 实现了4级流水线32bit全加器,每一级进行8bit加法运算且带有流水线暂停和刷新。
    \item 使用四个always语句块形成四级流水线
    \item 第一级:做[7:0]位与进位位的加法操作,并将运算结果和操作数AB传给下一级。
    \item 第二级:做[15:8]位与进位位的加法操作,并将本级运算结果和操作数AB传给下一级。
    \item 第三级:做[23:16]位与进位位的加法操作,并将运算结果和操作数AB传给下一级。
    \item 第四级:做[31:24]位与进位位的加法操作,并将结果组合输出。
    \item 经仿真验证,如后图,所设计流水线全加器功能正常。
\end{enumerate}

\subsection{问题1:Verilog语法问题和vivado操作问题}
\textbf{问题描述:}由于较长时间未使用Verilog语言和vivado,导致语法和软件操作生疏,具体有无符号拓展时语法报错,仿真时出错。

\textbf{解决方案:}自行复习拼接操作符的使用规范,发现是忘了在最后加上大括号而报错;仿真时未将文件设为顶层而无法完成仿真操作。

\subsection{问题2:流水线加法器仿真图不正确}
\textbf{问题描述:}具体有两种情况,一是运算结果并不是在第四个周期后得出,而是该周期立即得出;二是各级流水线并未存储上一级流水线未参与计算的数据,导致每一级流水线操作数混乱,运算结果出错。

\textbf{解决方案:}第一种情况应该是没有流水线的效果,重新编写为带流水线的加法器;第二种情况因为没有存储各级流水线未参与计算的数据,导致每一级流水线都是加上最新的操作数的各数位,重现添加寄存器存储上一级流水线参与运算的操作数即可。