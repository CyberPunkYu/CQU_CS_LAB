\section{实验过程记录}

\subsection{datapath,controller流水线化}
\begin{enumerate}
    \item 在实验三datapath的基础上,添加中间寄存器使整个流水线分为取指、译码、执行、访存和写回五个部分
    \item 在此基础上,datapath 的基本通路已经形成,下面加入控制器部分。控制器部分与单周期相同,仍然由 main decoder 和 alu decoder 构成,但由于改为五级流水线后,每一个阶段所需要的控制信号仅为一部分,控制器产生信号的阶段为译码阶段,产生控制信号后,依次通过触发器传到下一阶段,若当前阶段需要的信号,则不需要继续传递到下一阶段
\end{enumerate}

\subsection{冒险的解决}
\begin{enumerate}
    \item 冒险分为:数据冒险(寄存器中的值还未写回到寄存器堆中,下一条指令已经需要从寄存器堆中读取数据)和控制冒险(下一条要执行的指令还未确定,就按照 PC 自增顺序执行了本不该执行的指令),结构冒险已用两个存储器解决
    \item R型指令中,将 alu 得到的结果直接推送到下一条指令的 execute 阶段,同理,后续所有的阶段均已有结果,可以向对应的阶段推送,而不需要等到回写后再进行读取,达到数据前推的目的
    \item lw指令中,在 memory阶段才能够从数据存储器读取数据,此时 and 指令已经完成 ALU 计算,无法进行数据前推。必须使流水线暂停,等待数据读取后,再前推到 execute 阶段
    \item 根据以上实现逻辑连线即可解决数据冒险问题
    \item 控制冒险是分支指令引起的冒险。在五级流水线当中,分支指令在第 4 阶段才能够决定是否跳转。在 regfile 输出后添加一个判断相等的模块,即可提前判断 beq
    \item 再上一步的基础上再增加数据前推和流水线暂停模块,即解决了控制冒险问题
\end{enumerate}

\subsection{问题}
\subsubsection{问题描述}
仿真图不正确,部分接口的值为高阻态
\subsubsection{问题解决}
由于整个datapath、hazard模块的值变量名不统一,代码编写时结构和数据的书写都不太有逻辑,不方便debug,所以重新改写代码,按照取指,译码,执行,访存,回写五部进行分区改写。再改写过程中发现部分连线混淆,导致仿真出错。重新改写后正常。